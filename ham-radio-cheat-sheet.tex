\documentclass[a4paper,11pt]{article}
\pagestyle{headings}
\usepackage[utf8]{inputenc}
\usepackage[T1]{fontenc}
\usepackage[polish]{babel}
\author{Phitherek\_ SO9PH}
\title{HAM Radio Cheat Sheet}
\begin{document}
\maketitle
\section{Krótki wstęp}
W tym dokumencie przedstawiam w skrócie wszelkie informacje potrzebne do przeprowadzenia zgodnie z zasadami łączności radiowej na pasmach amatorskich. Aby samodzielnie przeprowadzić łączność na tych pasmach należy posiadać pozwolenie radiowe. Istnieją kluby krótkofalarskie, w których można taką łączność przeprowadzić pod nadzorem operatora odpowiedzialnego.
\section{Pasma i bandplan}
W zakresie fal krótkich (KF, ang. HF (High Frequency)) oraz ultrakrótkich (UKF, ang. VHF (Very High Frequency, 2m) i UHF (Ultra High Frequency, 70 cm)) istnieją poszczególne pasma, które mają różne właściwości. W każdym paśmie istnieją wydzielone częstotliwości tworzące pasma amatorskie - tylko na tych pasmach można prowadzić łączność amatorską. W ramach tych częstotliwości ustalane są podzakresy i ich przeznaczenie - nazywamy to bandplanem. Bandplan nie jest formalnie wiążący - jest tylko zaleceniem - natomiast w praktyce wszyscy go przestrzegają. Ważniejsze części bandplanu przedstawiam w poniższej tabeli.
\begin{center}
\begin{tabular}{| c | c | p{8cm} |}
\hline
\textbf{Pasmo} & \textbf{Zakres częstotliwości} & \textbf{Przeznaczenie} \\ \hline
80 m & 3500-3580 kHz & CW i dane (szerokość pasma < 200 Hz) \\ \cline{2-3}
 & 3580-3600 kHz & CW, RTTY i dane (szerokość pasma < 500 Hz) \\ \cline{2-3}
 & 3600-3620 kHz & CW, RTTY, dane, transmisje testowe, głos i obraz \\ \cline{2-3}
 & 3620-3800 kHz & CW, głos i obraz (szerokość pasma < 3 kHz) \\ \hline
40 m & 7000-7040 kHz & CW i dane (szerokość pasma < 200 Hz) \\ \cline{2-3}
 & 7040-7050 kHz & CW, RTTY i dane (szerokość pasma < 500 Hz) \\ \cline{2-3}
 & 7050-7060 kHz & CW, RTTY, dane, transmisje testowe, głos i obraz \\ \cline{2-3}
 & 7060-7100 kHz & CW, głos i obraz (szerokość pasma < 3 kHz) \\ \cline{2-3}
 & 7100-7200 kHz & CW, głos i obraz (szerokość pasma < 3 kHz, od marca 2009) \\ \cline{2-3}
 & 7200-7300 kHz & CW, głos i obraz (szerokość pasma < 3 kHz, jako drugorzędne) \\ \hline
30 m & 10100-10140 kHz & CW i dane (szerokość pasma < 200 Hz) \\ \cline{2-3}
 & 10140-10150 kHz & CW, RTTY i dane (szerokość pasma < 500 Hz) \\ \hline
20 m & 14000-14070 kHz & CW i dane (szerokość pasma < 200 Hz) \\ \cline{2-3}
 & 14070-14099 kHz & CW, RTTY i dane (szerokość pasma < 500 Hz) \\ \cline{2-3}
 & 14100 kHz & Zarezerwowane dla beaconów \\ \cline{2-3}
 & 14101-14350 kHz & CW, głos i obraz (szerokość pasma < 3 kHz) \\ \hline
17 m & 18068-18095 kHz & CW i dane (szerokość pasma < 200 Hz) \\ \cline{2-3}
 & 18095-18109 kHz & CW, RTTY i dane (szerokość pasma < 500 Hz) \\ \cline{2-3}
 & 18110 kHz & Zarezerwowane dla beaconów \\ \cline{2-3}
 & 18111-18168 kHz & CW, głos i obraz (szerokość pasma < 3 kHz) \\ \hline
15 m & 21000-21070 kHz & CW i dane (szerokość pasma < 200 Hz) \\ \cline{2-3}
 & 21070-21110 kHz & CW, RTTY i dane (szerokość pasma < 500 Hz) \\ \cline{2-3}
 & 21110-21120 kHz & CW, RTTY, dane, BEZ SSB (szerokość pasma < 2.7 kHz) \\ \cline{2-3}
 & 21120-21149 kHz & CW, RTTY i dane (szerokość pasma < 500 Hz) \\ \cline{2-3}
 & 21150 kHz & Zarezerwowane dla beaconów \\ \cline{2-3}
 & 21151-21450 kHz & CW, RTTY, dane, transmisje testowe, głos i obraz \\ \hline
12 m & 24890-24915 kHz & CW i dane (szerokość pasma < 200 Hz) \\ \cline{2-3}
 & 24915-24929 kHz & CW, RTTY i dane (szerokość pasma < 500 Hz) \\ \cline{2-3}
 & 24930 kHz & Zarezerwowane dla beaconów \\ \cline{2-3}
 & 24931-24990 kHz & CW, głos i obraz (szerokość pasma < 3 kHz) \\ \hline
10 m & 28000-28070 kHz & CW i dane (szerokość pasma < 200 Hz) \\ \cline{2-3}
 & 28070-28190 kHz & CW, RTTY i dane (szerokość pasma < 500 Hz) \\ \cline{2-3}
 & 28191-28224 kHz & Zarezerwowane dla beaconów \\ \cline{2-3}
 & 28225-29200 kHz & CW, głos i obraz (szerokość pasma < 3 kHz) \\ \cline{2-3}
 & 29200-29300 kHz & CW, dane, pakiety, transmisja FM, głos i obraz (szerokość pasma < 20 kHz) \\ \cline{2-3}
 & 29300-29510 kHz & Zarezerwowane na link satelitarny \\ \cline{2-3}
 & 29510-29700 kHz & CW, głos i obraz (szerokość pasma < 3 kHz) \\ \hline
\end{tabular}
\end{center}
\begin{center}
\begin{tabular}{| c | c | p{8cm} |}
\hline
6 m & 50.000-50.100 MHz & CW \\ \cline{2-3}
 & 50.100-50.500 MHz & CW, dane, RTTY, głos i obraz \\ \cline{2-3}
 & 50.500-52.000 MHz & CW, dane, RTTY i głos \\ \hline
2 m & 144.000-144.150 MHz & CW \\ \cline{2-3}
 & 144.150-144.500 MHz & CW i głos \\ \cline{2-3}
 & 144.500-145.000 MHz & CW, dane, RTTY, głos i obraz \\ \cline{2-3}
 & 145.000-146.000 MHz & CW, dane, RTTY, głos i obraz \\ \hline
70 cm & 430.000-432.000 MHz & CW, dane, RTTY, głos, obraz i TV \\ \cline{2-3}
 & 432.000-432.100 MHz & CW \\ \cline{2-3}
 & 432.100-432.400 MHz & CW, głos \\ \cline{2-3}
 & 432.400-440.000 MHz & CW, dane, RTTY, dane pakietowe, głos, obraz i TV \\ \hline
\end{tabular}
\end{center}
\subsection{Krótko o typach transmisji}
\begin{itemize}
\item Transmisja CW - to po prostu telegrafia, czyli porozumiewanie się nadając kluczem kod Morse' a.
\item Transmisja RTTY - to dalekopis.
\item Dane - to jakakolwiek transmisja cyfrowa.
\item Pakiety - transmisja PacketRadio.
\item Głos - inaczej transmisja foniczna, wykorzystuje głos ludzki. Istnieją trzy jej główne typy - FM, AM i wstęgowa (SSB), która dzieli się na dolnowstęgową (LSB) i górnowstęgową (USB).
\item Obraz - transmisja obrazu przez SSTV.
\end{itemize}
\subsection{Uwagi do bandplanu UKF}
Poniżej zamieszczam szczegółowe zastosowania zakresów częstotliwości w pasmach UKF.
\begin{center}
\begin{tabular}{| c | c | p{8cm} |}
\hline
\textbf{Pasmo} & \textbf{Zakres częstotliwości} & \textbf{Przeznaczenie} \\ \hline
2 m & 144.150-144.399 MHz & SSB \\ \cline{2-3}
 & 144.400-144.491 MHz & Radiolatarnie \\ \cline{2-3}
 & 145.000-145.194 MHz & Wejścia przemienników amatorskich \\ \cline{2-3}
 & 145.206-145.594 MHz & Simplex transmisji FM \\ \cline{2-3}
 & 145.600-145.794 MHz & Wyjścia przemienników amatorskich \\ \cline{2-3}
 & 145.800-146.000 MHz & Pasmo łączności satelitarnej \\ \hline
70 cm & 431.050-431.825 MHz & Wejścia przemienników amatorskich \\ \cline{2-3}
 & 432.400-432.490 MHz & Radiolatarnie \\ \cline{2-3}
 & 435.000-438.000 MHz & Pasmo łączności satelitarnej \\ \cline{2-3}
 & 438.650-439.425 MHz & Wyjścia przemienników amatorskich \\ \hline
\end{tabular}
\end{center}
\section{Znaki, literowanie i łamanie}
W tej części opisuję w jaki sposób identyfikować się w łączności amatorskiej.
\subsection{Znaki krótkofalarskie}
Krótkofalarski znak wywoławczy złożony jest z dwóch znaków alfanumerycznych (liter lub cyfr) zwanych prefiksem, następnie numerem, a następnie kolejnymi znakami alfanumerycznymi zwanymi sufiksem, których ilość jest zależna od przepisów danego kraju. W Polsce długość sufiksu ma od 1 do 4 znaków (stacje okolicznościowe do 7 znaków). Przykłady znaków to SQ9O, SO9PH, SQ9WTF lub SP9HACK. Znak krótkofalarski służy do jednoznacznej identyfikacji nadającej stacji.
\end{document}