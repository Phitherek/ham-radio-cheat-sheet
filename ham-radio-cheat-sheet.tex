\documentclass[a4paper,11pt]{article}
\usepackage[utf8]{inputenc}
\usepackage[T1]{fontenc}
\author{Phitherek\_ SO9PH}
\title{HAM Radio Cheat Sheet}
\begin{document}
\maketitle
\section{Krótki wstęp}
W tym dokumencie przedstawiam w skrócie wszelkie informacje potrzebne do przeprowadzenia zgodnie z zasadami łączności radiowej na pasmach amatorskich. Aby samodzielnie przeprowadzić łączność na tych pasmach należy posiadać pozwolenie radiowe. Istnieją kluby krótkofalarskie, w których można taką łączność przeprowadzić pod nadzorem operatora odpowiedzialnego.
\section{Pasma i bandplan}
W zakresie fal krótkich (KF, ang. HF (High Frequency)) oraz ultrakrótkich (UKF, ang. VHF (Very High Frequency, 2m) i UHF (Ultra High Frequency, 70 cm)) istnieją poszczególne pasma, które mają różne właściwości. W każdym paśmie istnieją wydzielone częstotliwości tworzące pasma amatorskie - tylko na tych pasmach można prowadzić łączność amatorską. W ramach tych częstotliwości ustalane są podzakresy i ich przeznaczenie - nazywamy to bandplanem. Bandplan nie jest formalnie wiążący - jest tylko zaleceniem - natomiast w praktyce wszyscy go przestrzegają. Ważniejsze części bandplanu w popularniejszych pasmach przedstawiam w poniższej tabeli.
\end{document}